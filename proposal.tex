\documentclass{article}%
\usepackage{amsmath}%
\usepackage{amsfonts}%
\usepackage{amssymb}%
\usepackage{graphicx}
\usepackage{cite}

%-------------------------------------------
\newtheorem{theorem}{Theorem}
\newtheorem{acknowledgement}[theorem]{Acknowledgement}
\newtheorem{algorithm}[theorem]{Algorithm}
\newtheorem{axiom}[theorem]{Axiom}
\newtheorem{case}[theorem]{Case}
\newtheorem{claim}[theorem]{Claim}
\newtheorem{conclusion}[theorem]{Conclusion}
\newtheorem{condition}[theorem]{Condition}
\newtheorem{conjecture}[theorem]{Conjecture}
\newtheorem{corollary}[theorem]{Corollary}
\newtheorem{criterion}[theorem]{Criterion}
\newtheorem{definition}[theorem]{Definition}
\newtheorem{example}[theorem]{Example}
\newtheorem{exercise}[theorem]{Exercise}
\newtheorem{lemma}[theorem]{Lemma}
\newtheorem{notation}[theorem]{Notation}
\newtheorem{problem}[theorem]{Problem}
\newtheorem{proposition}[theorem]{Proposition}
\newtheorem{remark}[theorem]{Remark}
\newtheorem{solution}[theorem]{Solution}
\newtheorem{summary}[theorem]{Summary}
\newenvironment{proof}[1][Proof]{\textbf{#1.} }{\ \rule{0.5em}{0.5em}}
\setlength{\textwidth}{7.0in}
\setlength{\oddsidemargin}{-0.35in}
\setlength{\topmargin}{-0.5in}
\setlength{\textheight}{9.0in}
\setlength{\parindent}{0.3in}

\newcommand{\kt}[1]{|#1\rangle}



\begin{document}



\begin{flushright}
    \textbf{Hongtao Zhang, Zhilin Du, Boyuan Zou \\}
\end{flushright}

\begin{center}
    \textbf{CS 880: Quantum Algorithm} \\
\end{center}

\section*{Quantum Monte Carlo}

\subsection*{Tentative Topics}

\begin{enumerate}
    \item Mean Estimation \cite{kothari2023mean, hamoudi2021quantum, montanaro2015quantum, HEINRICH20021}
    \item Quantum Monte Carlo and Many-body Problem \cite{ceperley1986quantum, boulder2003lecture}
    \item Complexity of Negative Sign Problem (Quantum Monte Carlo) \cite{troyer2005computational}
    \item Interaction between Quantum Monte Carlo and Quantum Computing \cite{huggins2022unbiasing, zhang2022quantum}
\end{enumerate}

\section*{Planned Outline}

We will firstly describe what is the classical Monte Carlo and what is quantum Monte Carlo.
Then we will describe quantum algorithm for classical Monte Carlo (which is the one published
in QIP 2023), and describe some quantum algorithm for quantum Monte Carlo.

We plan to follow the structure of lecture note from Yale \cite{boulder2003lecture} to learn and describe how Quantum Monte Carlo works.

We also tentatively plan to introduce the computational Complexity of a significant problem in 
Quantum Monte Carlo, the Negative Sign Problem, which is NP Hard \cite{troyer2005computational}.

We don't think we can actually tackle all of them, but the plan is to at least talk about 
the classical Monte Carlo and some introduction toward Quantum Monte Carlo.

\section*{Lower bound of Checking sort}

Checking if an array is sorted cost $O(N)$ for the classical method. 
We assume using quantum algorithm could reduce the complexity to $O(\sqrt{N})$. 
For this propose we want to design an algorithm to check if an array is sorted and find its corresponding complexity.
However, if we use a black box that could distinguish that, then this algorithm should be really similar as the phase kickback.
\\

Reference:
\begin{enumerate}
    \item Lower bound for Quantum sorting \cite{shi2000quantum}
    \item Quantum complexities of ordered searching, sorting, and element distinctness \cite{hoyer2002quantum}
\end{enumerate}


\section*{Quantum Machine Learning (Reinforcement Learning)}

We propose to present some Reinforcement Learning Theory Improvement 
based on Quantum Computing.

\subsection*{Tentative Topics}

\begin{enumerate}
    \item Reinforcement Learning Model \cite{agarwal2019reinforcement, dong2008quantum}.
    \item Quantum Reinforcement Learning \cite{dong2008quantum, meyer2022survey}.
    \item Example Maze Problem \cite{mackeprang2020reinforcement}.
    \item Quantum Deep Reinforcement Learning \cite{jerbi2021quantum}.
    \item Quantum simulated annealing \cite{jerbi2021quantum}.
\end{enumerate}

We plan to derive a paper that talks about Reinforcement Learning Model,
and how do people transform the classical Reinforcement Learning Model into
Quantum Reinforcement Learning Model, and how much speed benefit might be gained.

There are some tentative topics that we plan to include specifically in the project.
We plan to at least introduce the classical RL Model, and a few different quantum
adaptation toward the classical RL Model. Further, there are some algorithms
that we might be able to include, such as Quantum Simulated Annealing,
stemming from an old heuristic algorithm Simulated Annealing.
Further, we have included a paper specifically about Maze Problem,
which is generally the simplest problem the people discuss when 
talking about reinforcement learning.

If possible, we really want to talk about Quantum Deep Reinforcement Learning;
However, that might be too much given the time and our knowledge base.

\bibliographystyle{unsrt}

\bibliography{}

\end{document}