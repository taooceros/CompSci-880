\documentclass{article}%
\usepackage{amsmath}%
\usepackage{amsfonts}%
\usepackage{amssymb}%
\usepackage{graphicx}
\usepackage{cite}

%-------------------------------------------
\newtheorem{theorem}{Theorem}
\newtheorem{acknowledgement}[theorem]{Acknowledgement}
\newtheorem{algorithm}[theorem]{Algorithm}
\newtheorem{axiom}[theorem]{Axiom}
\newtheorem{case}[theorem]{Case}
\newtheorem{claim}[theorem]{Claim}
\newtheorem{conclusion}[theorem]{Conclusion}
\newtheorem{condition}[theorem]{Condition}
\newtheorem{conjecture}[theorem]{Conjecture}
\newtheorem{corollary}[theorem]{Corollary}
\newtheorem{criterion}[theorem]{Criterion}
\newtheorem{definition}[theorem]{Definition}
\newtheorem{example}[theorem]{Example}
\newtheorem{exercise}[theorem]{Exercise}
\newtheorem{lemma}[theorem]{Lemma}
\newtheorem{notation}[theorem]{Notation}
\newtheorem{problem}[theorem]{Problem}
\newtheorem{proposition}[theorem]{Proposition}
\newtheorem{remark}[theorem]{Remark}
\newtheorem{solution}[theorem]{Solution}
\newtheorem{summary}[theorem]{Summary}
\newenvironment{proof}[1][Proof]{\textbf{#1.} }{\ \rule{0.5em}{0.5em}}
\setlength{\textwidth}{7.0in}
\setlength{\oddsidemargin}{-0.35in}
\setlength{\topmargin}{-0.5in}
\setlength{\textheight}{9.0in}
\setlength{\parindent}{0.3in}

\newcommand{\kt}[1]{|#1\rangle}



\begin{document}



\begin{flushright}
    \textbf{Hongtao Zhang, Zhilin Du, Boyuan Zou \\}
\end{flushright}

\begin{center}
    \textbf{CS 880: Quantum Algorithm} \\
\end{center}

\section{Multidimensional Quantum Walks}

Tentative Topics

\begin{enumerate}
    \item Classical Random Walks
    \item Multidimensional Random Walks \cite[p.~295-299]{durrett2019probability}
          \begin{enumerate}
              \item Null-Recurrent Random Walk for $d\leqslant 2$
              \item Non-Recurrent Random Walk for $d\geqslant 3$
          \end{enumerate}
    \item Electric Network Framework \cite[p.~17]{jeffery2022multidimensional} \cite{belovs2013quantum} (briefly)
    \item Multidimensional Quantum Walks \cite[p.~]{jeffery2022multidimensional}
          \begin{enumerate}
              \item Alternating Neiborhoods
              \item Edge Composition
          \end{enumerate}
    \item Welded Tree \cite{jeffery2022multidimensional}
          \begin{enumerate}
              \item Welded Trees Problem
              \item Previous Work \cite{Childs_2003}
              \item Algorithm (exponentially speed up)
          \end{enumerate}
    \item k-distinctness \cite{jeffery2022multidimensional} (if we have time)
\end{enumerate}

We plan to cover some interesting topics about classical Multidimensional Random Walk
(about whether Multidimensional Random Walk is transient or (null) recurrent),
which is might be a bit out of topic, but likely be interesting.

Besides that, we will focus on the Multidimensional Quantum Walks paper itself.
Firstly we would like to talk about their framework and the framework they
are based on (Electric Network Framework). Then we will talk about their framework,
and the technique that are majorly discussed in the talk (Alternating Neiborhoods
and Edge Composition).

Finally, we will talk about the Welded Tree Problem, and how the
Multidimensional Quantum Walk may present an exponential speed up.

If we have time, we might also talk about the previous work about Welded Tree Problem
(according to their talk it involves continuous quantum walk).

They also discussed k-distinctness problem, and how the time complexity bound
get improved to match the query complexity. We might want to talk about if we have time.


\bibliographystyle{unsrt}

\bibliography{}

\end{document}