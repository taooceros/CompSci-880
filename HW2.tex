\documentclass{article}%
\usepackage{amsmath}%
\usepackage{amsfonts}%
\usepackage{amssymb}%
\usepackage{graphicx}
%-------------------------------------------
\newtheorem{theorem}{Theorem}
\newtheorem{acknowledgement}[theorem]{Acknowledgement}
\newtheorem{algorithm}[theorem]{Algorithm}
\newtheorem{axiom}[theorem]{Axiom}
\newtheorem{case}[theorem]{Case}
\newtheorem{claim}[theorem]{Claim}
\newtheorem{conclusion}[theorem]{Conclusion}
\newtheorem{condition}[theorem]{Condition}
\newtheorem{conjecture}[theorem]{Conjecture}
\newtheorem{corollary}[theorem]{Corollary}
\newtheorem{criterion}[theorem]{Criterion}
\newtheorem{definition}[theorem]{Definition}
\newtheorem{example}[theorem]{Example}
\newtheorem{exercise}[theorem]{Exercise}
\newtheorem{lemma}[theorem]{Lemma}
\newtheorem{notation}[theorem]{Notation}
\newtheorem{problem}[theorem]{Problem}
\newtheorem{proposition}[theorem]{Proposition}
\newtheorem{remark}[theorem]{Remark}
\newtheorem{solution}[theorem]{Solution}
\newtheorem{summary}[theorem]{Summary}
\newenvironment{proof}[1][Proof]{\textbf{#1.} }{\ \rule{0.5em}{0.5em}}
\setlength{\textwidth}{7.0in}
\setlength{\oddsidemargin}{-0.35in}
\setlength{\topmargin}{-0.5in}
\setlength{\textheight}{9.0in}
\setlength{\parindent}{0.3in}
\begin{document}

\newcommand{\ket}[1]{|#1\rangle}

\begin{flushright}
    \textbf{Hongtao Zhang \\}
\end{flushright}

\begin{center}
    \textbf{CS 880: Quantum Algorithm \\
        Homework NUM: 1} \\
\end{center}

\section*{Solution}

\section*{Question 1}

Because C-NOT gate requires check whether the control qubit is in state $|1\rangle$ or not,
so it will make the superposition of the control qubit collapse.

Therefore, it has the same effect as measure the 5 qubits as the for $z$ as the measurement in $y$.

\section*{Question 2}

It changes because of measurement collapse the distribution into a single state.

Therefore, the potential interference between the two qubits is destroyed, but instead become a linear combination as probability model.



\begin{align*}
    |00\rangle &= \begin{pmatrix}
                     1 \\
                     0 \\
                     0 \\
                     0 \\
                 \end{pmatrix} \\
    |01\rangle &= \begin{pmatrix}
                     0 \\
                     1 \\
                     0 \\
                     0 \\
                 \end{pmatrix} \\
    |10\rangle &= \begin{pmatrix}
                     0 \\
                     0 \\
                     1 \\
                     0 \\
                 \end{pmatrix} \\
    |11\rangle &= \begin{pmatrix}
                     0 \\
                     0 \\
                     0 \\
                     1\\
                 \end{pmatrix}\\
    \text{CNOT} &= \begin{pmatrix}
                1 & 0 & 0 & 0 \\
                0 & 1 & 0 & 0 \\
                0 & 0 & 0 & 1 \\
                0 & 0 & 1 & 0
            \end{pmatrix}\\
\end{align*}


\begin{equation*}
    \ket{0} \implies \ket{00}
\end{equation*}


\end{document}