\documentclass{article}%
\usepackage{amsmath}%
\usepackage{amsfonts}%
\usepackage{amssymb}%
\usepackage{graphicx}
%-------------------------------------------
\newtheorem{theorem}{Theorem}
\newtheorem{acknowledgement}[theorem]{Acknowledgement}
\newtheorem{algorithm}[theorem]{Algorithm}
\newtheorem{axiom}[theorem]{Axiom}
\newtheorem{case}[theorem]{Case}
\newtheorem{claim}[theorem]{Claim}
\newtheorem{conclusion}[theorem]{Conclusion}
\newtheorem{condition}[theorem]{Condition}
\newtheorem{conjecture}[theorem]{Conjecture}
\newtheorem{corollary}[theorem]{Corollary}
\newtheorem{criterion}[theorem]{Criterion}
\newtheorem{definition}[theorem]{Definition}
\newtheorem{example}[theorem]{Example}
\newtheorem{exercise}[theorem]{Exercise}
\newtheorem{lemma}[theorem]{Lemma}
\newtheorem{notation}[theorem]{Notation}
\newtheorem{problem}[theorem]{Problem}
\newtheorem{proposition}[theorem]{Proposition}
\newtheorem{remark}[theorem]{Remark}
\newtheorem{solution}[theorem]{Solution}
\newtheorem{summary}[theorem]{Summary}
\newenvironment{proof}[1][Proof]{\textbf{#1.} }{\ \rule{0.5em}{0.5em}}
\setlength{\textwidth}{7.0in}
\setlength{\oddsidemargin}{-0.35in}
\setlength{\topmargin}{-0.5in}
\setlength{\textheight}{9.0in}
\setlength{\parindent}{0.3in}
\begin{document}

\newcommand{\ket}[1]{|#1\rangle}

\begin{flushright}
    \textbf{Hongtao Zhang \\}
\end{flushright}

\begin{center}
    \textbf{CS 880: Quantum Algorithm \\
        Homework NUM: 3} \\
\end{center}

\section*{Solution}

\subsection{a}


\begin{enumerate}
    \item \begin{align*}
              \frac{1}{2} \begin{pmatrix}
                              1 \\
                              0 \\
                              0 \\
                              1
                          \end{pmatrix}\begin{pmatrix}
                                           1 & 0 & 0 & 1
                                       \end{pmatrix} & =1/2 \begin{pmatrix}
                                                                1 & 0 & 0 & 1 \\
                                                                0 & 0 & 0 & 0 \\
                                                                0 & 0 & 0 & 0 \\
                                                                1 & 0 & 0 & 1
                                                            \end{pmatrix} \\
          \end{align*}
    \item We know that
          \begin{align*}
              \rho  & = \frac{1}{2} \begin{pmatrix}
                                        1 & 0 & 0 & 1 \\
                                        0 & 0 & 0 & 0 \\
                                        0 & 0 & 0 & 0 \\
                                        1 & 0 & 0 & 1
                                    \end{pmatrix} \\
              \rho' & = \sum_{s}^{} P_s \rho P_s   \\
                    & = \frac{1}{2} \begin{pmatrix}
                                        1 & 0 & 0 & 0 \\
                                        0 & 1 & 0 & 0 \\
                                        0 & 0 & 0 & 0 \\
                                        0 & 0 & 0 & 0
                                    \end{pmatrix}
              \begin{pmatrix}
                  1 & 0 & 0 & 1 \\
                  0 & 0 & 0 & 0 \\
                  0 & 0 & 0 & 0 \\
                  1 & 0 & 0 & 1
              \end{pmatrix}
              \begin{pmatrix}
                  1 & 0 & 0 & 0 \\
                  0 & 1 & 0 & 0 \\
                  0 & 0 & 0 & 0 \\
                  0 & 0 & 0 & 0
              \end{pmatrix}
              + \frac{1}{2} \begin{pmatrix}
                                0 & 0 & 0 & 0 \\
                                0 & 0 & 0 & 0 \\
                                0 & 0 & 1 & 0 \\
                                0 & 0 & 0 & 1
                            \end{pmatrix}
              \begin{pmatrix}
                  1 & 0 & 0 & 1 \\
                  0 & 0 & 0 & 0 \\
                  0 & 0 & 0 & 0 \\
                  1 & 0 & 0 & 1
              \end{pmatrix}
              \begin{pmatrix}
                  0 & 0 & 0 & 0 \\
                  0 & 0 & 0 & 0 \\
                  0 & 0 & 1 & 0 \\
                  0 & 0 & 0 & 1
              \end{pmatrix}                       \\
                    & = \frac{1}{2} \begin{pmatrix}
                                        1 & 0 & 0 & 0 \\
                                        0 & 0 & 0 & 0 \\
                                        0 & 0 & 0 & 0 \\
                                        0 & 0 & 0 & 0
                                    \end{pmatrix}
              + \frac{1}{2} \begin{pmatrix}
                                0 & 0 & 0 & 0 \\
                                0 & 0 & 0 & 0 \\
                                0 & 0 & 0 & 0 \\
                                0 & 0 & 0 & 1
                            \end{pmatrix}         \\
                    & = \frac{1}{2} \begin{pmatrix}
                                        1 & 0 & 0 & 0 \\
                                        0 & 0 & 0 & 0 \\
                                        0 & 0 & 0 & 0 \\
                                        0 & 0 & 0 & 1
                                    \end{pmatrix}
          \end{align*}

    \item Base on the preset, we know that there are only two possible state (00) and (11),
          which means once we have measured the first bit, the second bit is determined, so the final measurement won't change anything.

          \begin{align*}
              \rho'' & = \sum_{s}^{} P_s \rho' P_s  \\
                     & = \frac{1}{2}
              \begin{pmatrix}
                  1 & 0 & 0 & 0 \\
                  0 & 0 & 0 & 0 \\
                  0 & 0 & 1 & 0 \\
                  0 & 0 & 0 & 0
              \end{pmatrix}
              \begin{pmatrix}
                  1 & 0 & 0 & 0 \\
                  0 & 0 & 0 & 0 \\
                  0 & 0 & 0 & 0 \\
                  0 & 0 & 0 & 1
              \end{pmatrix}
              \begin{pmatrix}
                  1 & 0 & 0 & 0 \\
                  0 & 0 & 0 & 0 \\
                  0 & 0 & 1 & 0 \\
                  0 & 0 & 0 & 0
              \end{pmatrix}
              + \frac{1}{2}
              \begin{pmatrix}
                  0 & 0 & 0 & 0 \\
                  0 & 1 & 0 & 0 \\
                  0 & 0 & 0 & 0 \\
                  0 & 0 & 0 & 1
              \end{pmatrix}
              \begin{pmatrix}
                  1 & 0 & 0 & 0 \\
                  0 & 0 & 0 & 0 \\
                  0 & 0 & 0 & 0 \\
                  0 & 0 & 0 & 1
              \end{pmatrix}
              \begin{pmatrix}
                  0 & 0 & 0 & 0 \\
                  0 & 1 & 0 & 0 \\
                  0 & 0 & 0 & 0 \\
                  0 & 0 & 0 & 1
              \end{pmatrix}                        \\
                     & = \frac{1}{2} \begin{pmatrix}
                                         1 & 0 & 0 & 0 \\
                                         0 & 0 & 0 & 0 \\
                                         0 & 0 & 0 & 0 \\
                                         0 & 0 & 0 & 1
                                     \end{pmatrix}
          \end{align*}
\end{enumerate}

\subsection{2}

\begin{enumerate}
    \item \begin{align*}
              \rho & = \frac{1}{2} \begin{pmatrix}
                                       1 & 1
                                   \end{pmatrix}\begin{pmatrix}
                                                    1 \\
                                                    1
                                                \end{pmatrix} \\
                   & = \frac{1}{2} \begin{pmatrix}
                                       1 & 1 \\
                                       1 & 1
                                   \end{pmatrix}              \\
          \end{align*}
    \item Because Bob has been measured, because of the information, alice will have the same state as bob, so alice will implicitly be measured.
          \begin{align*}
              \rho' & = \sum_{s}^{} P_s \rho P_s \\
                    & = \frac{1}{2}
              \begin{pmatrix}
                  1 & 0 \\
                  0 & 0
              \end{pmatrix}
              \begin{pmatrix}
                  1 & 1 \\
                  1 & 1
              \end{pmatrix}
              \begin{pmatrix}
                  1 & 0 \\
                  0 & 0
              \end{pmatrix}
          \end{align*} +
          \begin{align*}
              \frac{1}{2}
              \begin{pmatrix}
                  0 & 0 \\
                  0 & 1
              \end{pmatrix}
              \begin{pmatrix}
                  1 & 1 \\
                  1 & 1
              \end{pmatrix}
              \begin{pmatrix}
                  0 & 0 \\
                  0 & 1
              \end{pmatrix}                  \\
               & = \frac{1}{2} \begin{pmatrix}
                                   1 & 0 \\
                                   0 & 1
                               \end{pmatrix}
          \end{align*}
    \item After the final measure, alices should preserve the same behavior as (2) because of the situation we see in the entire system from (a)

\end{enumerate}


\end{document}