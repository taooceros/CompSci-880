\documentclass{article}%
\usepackage{amsmath}%
\usepackage{amsfonts}%
\usepackage{amssymb}%
\usepackage{graphicx}
%-------------------------------------------
\newtheorem{theorem}{Theorem}
\newtheorem{acknowledgement}[theorem]{Acknowledgement}
\newtheorem{algorithm}[theorem]{Algorithm}
\newtheorem{axiom}[theorem]{Axiom}
\newtheorem{case}[theorem]{Case}
\newtheorem{claim}[theorem]{Claim}
\newtheorem{conclusion}[theorem]{Conclusion}
\newtheorem{condition}[theorem]{Condition}
\newtheorem{conjecture}[theorem]{Conjecture}
\newtheorem{corollary}[theorem]{Corollary}
\newtheorem{criterion}[theorem]{Criterion}
\newtheorem{definition}[theorem]{Definition}
\newtheorem{example}[theorem]{Example}
\newtheorem{exercise}[theorem]{Exercise}
\newtheorem{lemma}[theorem]{Lemma}
\newtheorem{notation}[theorem]{Notation}
\newtheorem{problem}[theorem]{Problem}
\newtheorem{proposition}[theorem]{Proposition}
\newtheorem{remark}[theorem]{Remark}
\newtheorem{solution}[theorem]{Solution}
\newtheorem{summary}[theorem]{Summary}
\newenvironment{proof}[1][Proof]{\textbf{#1.} }{\ \rule{0.5em}{0.5em}}
\setlength{\textwidth}{7.0in}
\setlength{\oddsidemargin}{-0.35in}
\setlength{\topmargin}{-0.5in}
\setlength{\textheight}{9.0in}
\setlength{\parindent}{0.3in}

\newcommand{\kt}[1]{|#1\rangle}

\begin{document}

\section*{hw9}

\subsection*{Problem 1}

We know $\sin(\theta_0)^2=p\implies \sin(\theta_0)=\sqrt{p}$

Every iteration we add the angle by $\theta_0^2$, which means the success probability is a sequence

\[
    \sin(\theta_0)^2, \sin(3\theta_0)^2, \sin(5\theta_0)^2 \cdots
\]

From wolframalpha, the result

\[
    (1-\sin(\theta)^2)(1-\sin(3\theta)^2)\cdots(1-\sin((2k-1)\theta)^2
\]

is

\[
    4^{-k - 1} e^{-2 i \theta (k + 1)^2} (e^{-2 i \theta})^{2 k} (e^{4 i \theta})^k ((-e^{2 i \theta};e^{4 i \theta})_{k + 1})^2
\]
where $\theta=\sin^{-1}(\sqrt{p})$ and $(a;q)_n$ is the Pochhammer symbol.

Approximate the $\sin(\theta)\approx \theta$

we can get
\[
(1-p)(1-9p)(1-81p)\cdots (1-(2k-1)^2p)= (4^kp^k \Gamma(\frac{k+1}{2})^2)/\pi
\]

which grows in $p$.

\subsection*{Problem 2}

With a similar approximation, we can get

\[
    \sum\limits_{n=1}^{\infty} n\left(\prod_{i=1}^{n-1}(1-(2i-1)p)\right)(2n-1)p
\]
which should be approximately a geometric grow in order $p$, so the expected number of iterations is approximately $O(1/p)$.

\end{document}