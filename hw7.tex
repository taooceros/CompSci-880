\documentclass{article}%
\usepackage{amsmath}%
\usepackage{amsfonts}%
\usepackage{amssymb}%
\usepackage{graphicx}
%-------------------------------------------
\newtheorem{theorem}{Theorem}
\newtheorem{acknowledgement}[theorem]{Acknowledgement}
\newtheorem{algorithm}[theorem]{Algorithm}
\newtheorem{axiom}[theorem]{Axiom}
\newtheorem{case}[theorem]{Case}
\newtheorem{claim}[theorem]{Claim}
\newtheorem{conclusion}[theorem]{Conclusion}
\newtheorem{condition}[theorem]{Condition}
\newtheorem{conjecture}[theorem]{Conjecture}
\newtheorem{corollary}[theorem]{Corollary}
\newtheorem{criterion}[theorem]{Criterion}
\newtheorem{definition}[theorem]{Definition}
\newtheorem{example}[theorem]{Example}
\newtheorem{exercise}[theorem]{Exercise}
\newtheorem{lemma}[theorem]{Lemma}
\newtheorem{notation}[theorem]{Notation}
\newtheorem{problem}[theorem]{Problem}
\newtheorem{proposition}[theorem]{Proposition}
\newtheorem{remark}[theorem]{Remark}
\newtheorem{solution}[theorem]{Solution}
\newtheorem{summary}[theorem]{Summary}
\newenvironment{proof}[1][Proof]{\textbf{#1.} }{\ \rule{0.5em}{0.5em}}
\setlength{\textwidth}{7.0in}
\setlength{\oddsidemargin}{-0.35in}
\setlength{\topmargin}{-0.5in}
\setlength{\textheight}{9.0in}
\setlength{\parindent}{0.3in}

\newcommand{\kt}[1]{|#1\rangle}

\begin{document}

\section*{HW7}

\subsection*{1}

We can implement the addition part of the dot produce with CNOT gate,
but for the multiplication part, we might need a CCNOT gate.

There should be no way of simulating CCNOT with plain CNOT,
so we will need to find some other way of implementing the multiplication.

The major issue with CNOT and not is it is not possible to distinguish between
state $\kt{00}$ and $\kt{11}$, while we only want to add if it is $\kt{11}$,
which says it always have two states that will result in positive, and two negative.

For single qubit ($k=1$), it is exactly a CCNOT gate.
At least we need a CCNOT gate to implement, but it seems like that's not possible.


\subsection*{2}

We can just do a matrix multiplication and see the results is correct.

\[
    \begin{pmatrix}
        1 & 1  & 1  & 1  \\
        1 & -1 & 1  & -1 \\
        1 & 1  & -1 & -1 \\
        1 & -1 & -1 & 1
    \end{pmatrix}
    \begin{pmatrix}
        1 & 0 & 0 & 0 \\
        0 & 1 & 0 & 0 \\
        0 & 0 & 0 & 1 \\
        0 & 0 & 1 & 0
    \end{pmatrix}
    \begin{pmatrix}
        1 & 1  & 1  & 1  \\
        1 & -1 & 1  & -1 \\
        1 & 1  & -1 & -1 \\
        1 & -1 & -1 & 1
    \end{pmatrix}
    =
    \begin{pmatrix}
        1 & 0 & 0 & 0 \\
        0 & 0 & 0 & 1 \\
        0 & 0 & 1 & 0 \\
        0 & 1 & 0 & 0
    \end{pmatrix}
\]

\subsection*{3}

Whenever we have a single cnot gate between the two Hadamard gate,
we can just inverse that cnot gate and split out the two Hadamard gate,
and remove the Hadamard gate corresponding to the cnot circuits.



\end{document}